\documentclass{beamer}
\usetheme{Dresden}
\usecolortheme{spruce}
\usepackage[utf8]{inputenc}
\usepackage[russian]{babel}
\usepackage{amsmath}
\title{Многочлены}
\subtitle{Введение в специальность}
\author{Арина Плюснина}
\institute{БФУ им. И. Канта}
\date{\today}
\begin{document}
\begin{frame}
	\titlepage
\end{frame}
\begin{frame}{Основные понятия}
	\textbf{Многочлен} - это выражение вида \begin{center}$f(T)=a_{n} T^n+a_{n-1}T^{n-1}+...+a_1T^1+a_0$,
	\end{center}
	где $a_i\in K$, $K$ - произвольное поле;
	$a_n$ -  старший коэффициент;
	$n=\deg f$ - степень многочлена.
	\newline $K[T]$ - множество многочленов от переменной $T$ с коэффициентами из поля $K$ - коммутативное кольцо.
	\newline \textbf{Теорема Безу:} $f(T)\in K[T]$ \textit{имеет корень} $\rho\in K$ \textit{тогда и только тогда, когда} $f(T)$ \textit{делится на} $T-\rho$.
	\newline \textbf{Замечание:} Количество корней $S$ многочлена $f(T)\in K[T]$ не превосходит степень $f(T)$: $S\le\deg f(T)$. В частности, если $S=\deg f(T)$ $\Rightarrow$ $f(T)=(T-\rho_1)\cdot...\cdot(T-\rho_s)\cdot a_s$, $a_s$ - старший коэффициент. 
\end{frame}

\begin{frame}{Основные понятия}
	Опр: $\rho\in K$ называют \textbf{корнем кратности} $k$ многочлена $f(T)$, если 
	\begin{center}
		$f(T)=(T-\rho)^kh(T)$, $h(T)\ne0$.
	\end{center}
	\textbf{Предложение:} $\sqsupset f(T)\in K[T]$, $\rho\in K$ \textit{есть кратный корень} $f(T)$ $\Leftrightarrow$ $f'(\rho)=0$
	\newline \textbf{Замечание:} рассмотрим $f(T)=aT^3+bT^2+cT+d$, $a,b,c,d\in K$, $a\ne0$
	\newline Теорема Виета для кубического уравнения:
	\begin{center}
		$\rho_1+\rho_2+\rho_3=-\frac{b}{a}$ \\$\rho_1\rho_2+\rho_2\rho_3+\rho_1\rho_3=\frac{c}{a}$ \\$\rho_1\rho_2\rho_3=-\frac{d}{a}$  
	\end{center}
	\end{frame}

\end{document}