\documentclass{beamer}
\usetheme{Dresden}
\usecolortheme{spruce}
\usepackage[utf8]{inputenc}
\usepackage[russian]{babel}
\usepackage{amsmath}
\title{Многочлены}
\subtitle{Введение в специальность}
\author{Арина Плюснина}
\institute{БФУ им. И. Канта}
\date{\today}
\begin{document}
	\begin{frame}
		\titlepage
	\end{frame}
	\begin{frame}{Основные понятия}
		\newtheorem{opr}{Определение}
		\newtheorem{theorm}{Теорема}
		\begin{opr}Многочлен - это выражение вида
			\[f(T)=a_{n} T^n+a_{n-1}T^{n-1}+...+a_1T^1+a_0,\]
			где $a_i\in K$, $K$ - произвольное поле;
			$a_n$ -  старший коэффициент;
			$n=\deg f$ - степень многочлена.
		\end{opr}
			\begin{theorm}[Теорема Безу] $f(T)\in K[T]$ имеет корень $\rho\in K$ тогда и только тогда, когда $f(T)$ делится на $T-\rho$.
			\end{theorm}
	\end{frame}
	
	\begin{frame}{Основные понятия}
		
		\begin{opr}
			$\rho\in K$ называют \textbf{корнем кратности} $k$ многочлена $f(T)$, если 
				\[f(T)=(T-\rho)^kh(T), h(T)\ne0.\]
		\end{opr}
		\begin{itemize}
			\item Теорема Виета для кубического уравнения:
			\begin{equation*}
			\begin{split}
			\rho_1+\rho_2+\rho_3&=-\frac{b}{a}
			\\\rho_1\rho_2+\rho_2\rho_3+\rho_1\rho_3&=\frac{c}{a}
			\\\rho_1\rho_2\rho_3&=-\frac{d}{a}
			\end{split}
			\end{equation*}
		\end{itemize}
	\end{frame}
\end{document}