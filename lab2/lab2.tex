\documentclass{beamer}
\usetheme{Dresden}
\usecolortheme{spruce}
\usepackage[utf8]{inputenc}
\usepackage[russian]{babel}
\title{Натуральные целые числа}
\subtitle{Введение в специальность}
\author{Арина Плюснина}
\institute{БФУ им. И. Канта}
\date{\today}
\begin{document}
\begin{frame}
	\titlepage
\end{frame}
\begin{frame}{Задачи}
	\begin{enumerate}
		\item Изучить основные понятия и теоремы
		\item Использовать их для дальнейшего изучения материала
	\end{enumerate}
\end{frame}
\begin{frame}
	\frametitle{План}
	\tableofcontents
	\section{Основные понятия}
	\subsection{Принцип минимальность}
	\subsection{Принцип индукции}
	\section{Начало Криптографии}
	\section{Целые числа}
	\subsection{Алгоритм Евклида}
	\subsection{Лемма Евклида}
	\subsection{Основная теорема арифметики}
	\section{Сравнения}
	\subsection{Определение и свойства}
	\subsection{Класс вычитов}
	\subsection{НОД}
	\section{Простейшие аффинные криптосистемы}
\end{frame}

\begin{frame}{Вывод}
	\begin{itemize}
	\item Тема была изучена
	\end{itemize}	
\end{frame}
\end{document}