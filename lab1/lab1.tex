\documentclass[11pt]{article}
\usepackage{amsmath,amssymb,amsthm}
\usepackage{algorithm}
\usepackage[noend]{algpseudocode} 
\usepackage{fancyhdr}
%---enable russian----

\usepackage[utf8]{inputenc}
\usepackage[russian]{babel}

% PROBABILITY SYMBOLS
\newcommand*\PROB\Pr 
\DeclareMathOperator*{\EXPECT}{\mathbb{E}}


% Sets, Rngs, ets 
\newcommand{\N}{{{\mathbb N}}}
\newcommand{\Z}{{{\mathbb Z}}}
\newcommand{\R}{{{\mathbb R}}}
\newcommand{\Zp}{\ints_p} % Integers modulo p
\newcommand{\Zq}{\ints_q} % Integers modulo q
\newcommand{\Zn}{\ints_N} % Integers modulo N

% Landau 
\newcommand{\bigO}{\mathcal{O}}
\newcommand*{\OLandau}{\bigO}
\newcommand*{\WLandau}{\Omega}
\newcommand*{\xOLandau}{\widetilde{\OLandau}}
\newcommand*{\xWLandau}{\widetilde{\WLandau}}
\newcommand*{\TLandau}{\Theta}
\newcommand*{\xTLandau}{\widetilde{\TLandau}}
\newcommand{\smallo}{o} %technically, an omicron
\newcommand{\softO}{\widetilde{\bigO}}
\newcommand{\wLandau}{\omega}
\newcommand{\negl}{\mathrm{negl}} 

% Misc
\newcommand{\eps}{\varepsilon}
\newcommand{\inprod}[1]{\left\langle #1 \right\rangle}

 
\newcommand{\handout}[5]{
  \noindent
 
  \vspace*{4mm}
}

\newcommand{\lecture}[4]{\handout{#1}{#2}{#3}{Scribe: #4}{Тема #1}}

\newtheorem{theorem}{Теорема}
\newtheorem{lemma}{Лемма}
\newtheorem{definition}{Определение}
\newtheorem{corollary}{Следствие}
\newtheorem{fact}{Факт}

% 1-inch margins
\topmargin 0pt
\advance \topmargin by -\headheight
\advance \topmargin by -\headsep
\textheight 8.9in
\oddsidemargin 0pt
\evensidemargin \oddsidemargin
\marginparwidth 0.5in
\textwidth 6.5in

\parindent 0in
\parskip 1.5ex


\pagestyle{fancy}
\fancyhead{}
\fancyhead[LE,RO]{\thepage}
\fancyhead[CO]{Теорема Вильсона}
\fancyhead[LO]{Глава 5} 
\fancyhead[CE]{Теорема Вильсона} 
\fancyfoot{}

\begin{document}

целые числа $2, 3, ..., 11$ в $(p-3)/2=5$ пар, каждое из произведений которых сравнимо с $1$ по модулю $13$. Запишем эти сравнения: 
\begin{equation*}
\begin{split}
	2\cdot7&\equiv 1\bmod 13,
	\\3\cdot9&\equiv 1\bmod 13,
	\\4\cdot10&\equiv 1\bmod 13,
	\\5\cdot8&\equiv 1\bmod 13,
	\\6\cdot11&\equiv 1\bmod 13.
\end{split}
\end{equation*}
Умножение вышеприведенных сравнений дает результат
	\[11! = (2\cdot7)(3\cdot9)(4\cdot10)(5\cdot8)(6\cdot11)\equiv1\bmod13\]
и также
	\[12!\equiv12\equiv-1\bmod13\]
Таким образом, $(p-1)!\equiv-1\bmod p$, где $p=13$.
\parindent Верно и обратное утверждение теоремы Вильсона: если $(n-1)!\equiv-1\bmod n$,то $n$ должно быть простым числом. Ибо, если $n$ не простое, то $n$ имеет делитель с $d$ такой, что $1<d<n$. Кроме того, поскольку $d\le n-1$, $d$ одно из коэффициентов в $(n-1)!$, где $d\mid(n-1)!$. Теперь мы предполагаем, что $n\mid(n-1)!+1$, и также $d\mid(n-1)!+1$. Вывод состоит в том, что $d\mid1$, что противоречит начальному утверждению.
Взятые вместе, теорема Вильсона и ее обратное утверждение составляют необходимое и достаточное условие для определения примитивности; а именно, целое $n>1$ является простым тогда и только тогда, когда $(n-1)!\equiv-1\bmod n$. К счастью, этот тест представляет скорее теоретический, чем практический интерес, т.к. по мере увеличения $n$, $(n-1)!$ быстро становится неуправляемым в размерах.
Мы хотели бы закончить эту главу применением теоремы Вильсона к изучения квадратичных сравнений.[Понятно, что \textit{квадратичное сравнение} означает сравнение вида $ax^2+bx+c\equiv0\bmod n$,где $a\neq0$.] Это и есть содержание
\begin{theorem}\label{task:b}
	Квадратичное сравнение $x^2+1\equiv 0 \bmod p$, где $p$ - простое число, имеет решение тогда и только тогда, когда $p\equiv 1\bmod 4$
\end{theorem}
\begin{proof}
	Пусть $a$ является решением $x^2+1\equiv p\bmod p$, тогда $a^2\equiv-1\bmod p$. Поскольку $p\nmid a$, результатом применения теоремы Ферма будет:
		\[1\equiv a^{p-1}\equiv (a^2)^{(p-1)/2}\equiv (-1)^{(p-1)/2}\bmod p.\]
Возможность того, что $p=4k+3$ для некоторого $k$ не возникает. Если бы это было так, мы бы это сделали
	\[(-1)^{(p-1)/2}=(-1)^{2k+1}=-1;\]
следовательно, $1\equiv-1\bmod p$. Конечным результатом этого является то, что $p\mid2$, которое является заведомо ложным. Следовательно, $p$ должен иметь вид $4k+1$.
Теперь в противоположном направлении. В выражении
	\[(p-1)!=1\cdot2\cdots\frac{p-1}{2}\cdot\frac{p+1}{2}\cdots(p-2)(p-1)\]
мы имеем следующие сравнения
\begin{equation*}
\begin{split}
	p-1&\equiv-1\bmod p,
	\\p-2&\equiv-2\bmod p,
	\\&\cdots
	\\\frac{p+1}{2}&\equiv-\frac{p-1}{2}\bmod p.
\end{split}
\end{equation*}
Перестановка коэффициентов приводит 
	\[(p-1)!=1\cdot(-1)\cdot2\cdot(-2)\cdots\frac{p-1}{2}\cdot(-\frac{p-1}{2})\bmod p\equiv(-1)^{(p-1)/2}(1\cdot2\cdots\frac{p-1}{2})^2\bmod p\]
т.к. здесь $(p-1)/2$ задействованы минусовые знаки. Именно в этот момент теорема Вильсона может быть применена; ибо, $(p-1)!\equiv-1\bmod p$, откуда 
	\[-1\equiv(-1)^{(p-1)/2}[(\frac{p-1}{2})!]^2\bmod p.\]
Если мы предположим, что $p$ имеет вид $4k+1$, то $(-1)^{(p-1)/2}=1$, оставляет нас со сравнением 
	\[-1\equiv[(\frac{p-1}{2})!]^2\bmod p.\]
Вывод: $[(p-1)/2]!$ удовлетворяет квадратичному сравнению $x^2+1\equiv 0 \bmod p$.
\end{proof}
\parindent Давайте рассмотрим реальный пример, скажем, случай $p=13$, который является простым числом вида $4k+1$. Здесь, мы имеем $(p-1)/2=6$ и легко увидеть, что 
	\[6!=720\equiv5\bmod13,\]
пока
	\[5^2+1=26\equiv0\bmod13.\]
Таким образом утверждение $[(\frac{1}{2}(p-1))!]^2+1\bmod p$ выполняется для $p=13$.
Теорема Вильсона подразумевает, что существует бесконечность составных чисел вида $n!+1$. С другой стороны, это открытый вопрос, является ли $n!+1$ простым числом для бесконечно многих значений $n$. Единственные значения в диапазоне $1\le n\le100$, для которых $n!-1$, как известно, является простым числом, $n=1,2,3,11,27,37,41,73$, и $77$. 
\paragraph{Задания 5.4}
\begin{enumerate}
	\item\begin{enumerate}
		\item Найдите остаток от деления $15!$ на $17$.
		\item Найдите остаток от деления $2(26!)$ на $29$. [\textit{Подсказка:} Из теоремы~\ref{task:b}, $2(p-3)!\equiv-1\bmod p$ для некоторого простого числа $p>3$.]
		\end{enumerate}
	\item Определите, является ли число $17$ простым, решив верно ли $16!\equiv-1\bmod 17$.
	\item Расположите числа $2$, $3$, $4$, ..., $21$ в пары вида $a$ и $b$ со свойством $ab\equiv-1\bmod23$.
	\item Покажите, что $18!\equiv-1\bmod437$.
	\item\begin{enumerate}
		\item Докажите, что целое число $n>1$ простое тогда и только тогда, когда $(n-2)!\equiv1\bmod n$.
		\item Если $n$ составное число, покажите, что $(n-1)!\equiv0\bmod n$, кроме $n=4$.
		\end{enumerate}
	\item Учитывая, что $p$ простое, установите сравнение
		\[(p-1)!\equiv p-1\bmod 1+2+3+\cdots+(p-1).\]
	\item Если $p$ простое число, докажите, что
		\[p\mid a^p+(p-1)!a \text{ и } p\mid(p-1)!a^p+a\]
	для некоторого целого $a$. [\textit{Подсказка:} Из теоремы~\ref{task:b}, $a^p+(p-1)!a\equiv a^p-a\bmod p$.]
	\item Найдите два простых числа $p\le13$, для которого сравнение $(p-1)!\equiv-1\mod p^2$ выполняется.
	\item Используя теорему~\ref{task:b}, докажите, что
		\[1^2\cdot3^2\cdot5^2\cdots(p-2)^2\equiv(-1)^{(p+1)/2}\bmod p\]
	для некоторого простого числа $p$. [\textit{Подсказка:} Поскольку $k\equiv-(p-k)\bmod p$, из этого следует, что $2\cdot4\cdot6\cdots(p-1)\equiv(-1)^{(p-1)/2}1\cdot3\cdot5\cdots(p-2)\bmod p$.]
	\item\begin{enumerate}
		\item\label{task:a} Для простого $p$ вида $4k+3$, докажите, что если
			\[(\frac{p-1}{2})!\equiv1\bmod p \text{ или } (\frac{p-1}{2})!\equiv-1\bmod p;\]
		следовательно, $[(p-1)/2]!$ удовлетворяет квадратичному сравнению $x^2\equiv1\bmod p$.
		\item Используйте~\ref{task:a}, чтобы показать, что если $p=4k+3$ является простым, то произведение всех простых чисел, меньших, чем $p$, сравнимы по модулю $p$ с $1$ или $-1$. [\textit{Подсказка:} Теорема Ферма подразумевает, что $2^{(p-1)/2}\equiv\pm1\bmod p$.]
		\end{enumerate}
	\item Примените теорему~\ref{task:b}, чтобы найти решения для квадратичных сравнений $x^2\equiv-1\bmod 29$ и $x^2\equiv-1\bmod37$
	\item Покажите, что если $p=4k+3$ является простым и $a^2+b^2\equiv0\mod p$, то $a\equiv b\equiv0\bmod p$. [\textit{Подсказка:} Если $a\ne0\bmod p$, то существует целое $c$ такое, что $ac\equiv1\bmod p$; используйте этот факт, чтобы противоречить теореме~\ref{task:b}.]
\end{enumerate}
\end{document}